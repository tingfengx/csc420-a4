\documentclass[11pt, letter]{article}
\usepackage{geometry}
\usepackage{datetime}
\usepackage[T1]{fontenc}
\usepackage{titling}
\usepackage{titlesec}
\usepackage{hyperref}
\usepackage{graphicx}
\usepackage{multicol}
\usepackage{soul}
\usepackage{tcolorbox}
\usepackage[font=footnotesize]{caption}
\usepackage{subcaption}
\usepackage{chngcntr}
\usepackage{float}
\usepackage{booktabs}
\usepackage{pgfplots}
\usepackage{amsmath}
\usepackage{amssymb}
\usepackage{amsfonts}
\usepackage[parfill]{parskip}
\captionsetup{
	format=hang, 
    singlelinecheck=false
}

% General
\newcommand{\mc}[1]{\mathcal{#1}}

% Math Bold Font, Vector Notations
\newcommand{\ba}{\mathbf{a}}
\newcommand{\bb}{\mathbf{b}}
\newcommand{\bc}{\mathbf{c}}
\newcommand{\bd}{\mathbf{d}}
\newcommand{\be}{\mathbf{e}}
\renewcommand{\bf}{\mathbf{f}}
\newcommand{\bg}{\mathbf{g}}
\newcommand{\bh}{\mathbf{h}}
\newcommand{\bi}{\mathbf{i}}
\newcommand{\bj}{\mathbf{j}}
\newcommand{\bk}{\mathbf{k}}
\newcommand{\bl}{\mathbf{l}}
\newcommand{\bm}{\mathbf{m}}
\newcommand{\bn}{\mathbf{n}}
\newcommand{\bo}{\mathbf{o}}
\newcommand{\bp}{\mathbf{p}}
\newcommand{\bq}{\mathbf{q}}
\newcommand{\br}{\mathbf{r}}
\newcommand{\bs}{\mathbf{s}}
\newcommand{\bt}{\mathbf{t}}
\newcommand{\bu}{\mathbf{u}}
\newcommand{\bv}{\mathbf{v}}
\newcommand{\bw}{\mathbf{w}}
\newcommand{\bx}{\mathbf{x}}
\newcommand{\by}{\mathbf{y}}
\newcommand{\bz}{\mathbf{z}}
\newcommand{\bzero}{\mathbf{0}}

% Proofs, Structures
\newcommand{\proof}{\tit{\underline{Proof:}}} % This equivalent to the \begin{proof}\end{proof} block
\newcommand{\proofforward}{\tit{\underline{Proof($\implies$):}}}
\newcommand{\proofback}{\tit{\underline{Proof($\impliedby$):}}}
\newcommand{\proofsuperset}{\tit{\underline{Proof($\supseteq$):}}}
\newcommand{\proofsubset}{\tit{\underline{Proof($\subseteq$):}}}
\newcommand{\contradiction}{$\longrightarrow\!\longleftarrow$}
\newcommand{\qed}{\hfill $\blacksquare$}

% Number Spaces, Vector Space
\newcommand{\R}{\mathbb{R}}
\newcommand{\real}{\mathbb{R}}
\newcommand{\complex}{\mathbb{C}}
\newcommand{\field}{\mathbb{F}}

% customized commands
\newcommand{\settag}[1]{\renewcommand{\theenumi}{#1}}
\newcommand{\tbf}[1]{\textbf{#1}}
\newcommand{\tit}[1]{\textit{#1}}
\newcommand{\overbar}[1]{\mkern 1.5mu\overline{\mkern-1.5mu#1\mkern-1.5mu}\mkern 1.5mu}
\newcommand{\double}[1]{\mathbb{#1}} % Set to behave like that on word
\newcommand{\trans}[3]{$#1:#2\rightarrow{}#3$}
\newcommand{\map}[3]{\text{$\left[#1\right]_{#2}^{#3}$}}
\newcommand{\dime}[1]{\mathrm{dim}(#1)}
\newcommand{\mat}[2]{M_{#1 \times #2}(\R)}
\newcommand{\aug}{\fboxsep=-\fboxrule\!\!\!\fbox{\strut}\!\!\!}
\newcommand{\basecase}{\textsc{\underline{Basis Case:}} }
\newcommand{\inductive}{\textsc{\underline{Inductive Step:}} }
\newcommand{\norm}[1]{\left\lVert#1\right\rVert}
\newcommand{\independent}{\perp \!\!\! \perp}

% Set section number
\counterwithin{equation}{section}
\counterwithin{footnote}{section}
\counterwithin{figure}{section}
\counterwithin{table}{section}

\titleformat{\section}[runin]
  {\normalfont\Large\bfseries}{\thesection}{1em}{}
\titleformat{\subsection}[runin]
  {\normalfont\large\bfseries}{\thesubsection}{1em}{}
  
\renewcommand{\thesection}{\Roman{section}} 
\renewcommand{\thesubsection}{\thesection.\Roman{subsection}}

\makeatletter

\renewcommand{\maketitle}{\bgroup\setlength{\parindent}{0pt}
\begin{flushleft}
  \textbf{\Large \@title}

  \@author\\
  Date: \@date
\end{flushleft}\egroup
}
\makeatother

\author{Author: Tingfeng Xia, University of Toronto}
\title{Assignment 4 \\ \small{CSC420 Introduction to Image Understanding}}

\date{\today}

\begin{document}
\maketitle

%\tableofcontents
%\section{Disclosure and Notes}
%\begin{itemize}
%	\item I used yada yada here
%\end{itemize}
\section{Deep Learning} 
We have the computation graph, written symbolically (with the corresponding variable names in the code \texttt{q1.py})
\begin{align}
	\texttt{sum1} &= \Sigma_1 = w_1x_1 + w_2x_2 \\
	\texttt{sum2} &= \Sigma_2 = w_3x_3 + w_4x_4 \\
	\texttt{sigma1} &= \sigma_1 = \sigma(\Sigma_1) \\
	\texttt{sigma2} &= \sigma_2 = \sigma(\Sigma_2) \\
	\texttt{sum3} &= \Sigma_3 = w_5\sigma_1 + w_6\sigma_2 \\
	\texttt{sigma3} &= \sigma_3 = \sigma(\Sigma_3) \\
	\texttt{yhat} &= \hat{y} = \sigma_3 \\
	\texttt{L} &= L = \norm{\hat{y} - y}_2^2
\end{align}
In \texttt{q1.py}, I calculated the forward pass and then back propagated the error signal. It prints the following result, including intermediate values: 
\begin{verbatim}
-> Initialization ... <-
x1 = 0.90000, x2 = -1.10000, x3 = -0.30000, x4 = 0.80000
w1 = 0.75000, w2 = -0.63000, w3 = 0.24000, 
w4 = -1.70000, w5 = 0.80000, w6 = -0.20000
target y = 0.50000
-> Start Forward Pass <-
sum1 = 1.36800, sigma1 = 0.79706, sum2 = -1.43200
sigma2 = 0.19279, sum3 = 0.59909, sigma3 = 0.64545
yhat = 0.64545, L = 0.02116
-> Start Back Propagation <-
dLdL = 1.00000, dLdyhat = 0.29090, dLdsigma3 = 0.29090
dLdsum3 = 0.06657, dLdsigma2 = -0.01331, dLdsum2 = -0.00207
-> End Result: dLdw3 = 0.0006215780
\end{verbatim}
Then, our desired final result is 
\begin{equation}
	\frac{\partial L}{\partial w_3} = .0006215780
\end{equation}

\section{Camera Models}
\subsection{Part 1} First we expand as hinted,
\begin{align}
	p &= \begin{bmatrix}
		wx \\ wy \\ w
	\end{bmatrix} = KP = K \begin{bmatrix}
		X_0 + td_x \\ Y_0 + td_y \\ Z_0 + td_z
	\end{bmatrix}
	= \begin{bmatrix}
		f & 0 & p_x \\
		0 & f & p_y \\
		0 & 0 & 1
	\end{bmatrix}\begin{bmatrix}
		X_0 + td_x \\ Y_0 + td_y \\ Z_0 + td_z
	\end{bmatrix} \\
	&= \begin{bmatrix}
		f(X_0 + td_x) + p_x (Z_0 + td_z) \\
		f(Y_0 + td_y) + p_y (Z_0 + td_z) \\
		Z_0 + td_z
	\end{bmatrix}
\end{align}
then we can solve for $x, y$ in where $p = (wx, wy, w)^\top$. We have
\begin{equation}
	x = \frac{wx}{w} = \frac{f(X_0 + td_x) + p_x (Z_0 + td_z)}{Z_0 + td_z}
\end{equation}
and 
\begin{equation}
	y = \frac{wy}{w} = \frac{f(Y_0 + td_y) + p_y (Z_0 + td_z)}{Z_0 + td_z}
\end{equation}
which are (parametric) coordinates for the line on the image plane. To find the pixel coordinates o the vanishing point corresponding to the line, it suffices to take limit of $t$ tends to infinity,
\begin{equation}
	x_v = \lim_{t\rightarrow\infty} \frac{f(X_0 + td_x) + p_x (Z_0 + td_z)}{Z_0 + td_z} = \lim_{t\rightarrow\infty} \frac{ftd_x + p_xtd_z}{td_z} = \frac{fd_x + p_xd_z}{d_z}
\end{equation}
and
\begin{equation}
	y_v = \lim_{t\rightarrow\infty} \frac{f(Y_0 + td_y) + p_y (Z_0 + td_z)}{Z_0 + td_z} = \lim_{t\rightarrow\infty} \frac{ftd_y + p_ytd_z}{td_z} = \frac{fd_y + p_yd_z}{d_z}
\end{equation}
Then, the vanishing point is, on the image plane, $(x_v, y_v)$, defined everywhere except for when $d_z = 0$, in which case vanishing point does not exist. 

\subsection{Part 2} What we have shown in Part 1 is true for all $d$. Now that we want to show for all lines that are on the plane, then it suffices to constraint $d$ such that $\langle n , d \rangle = 0$, where $n$ is the normal vector to the plane. This gives us an extra equation
$
	n_xd_x + n_yd_y + n_zd_z = 0
$
Our goal here is to show that $\alpha x_v + \beta y_v = $ constant that is independent of $d$ for some $\alpha, \beta$. (Notice that $x_v, y_v$ is already independent of $P$, the starting point.) We can start with the inner product constraint, which is
\begin{equation}
	n_xd_x + n_yd_y + n_zd_z = 0
\end{equation}
then since $n_zd_z \neq 0$ (otherwise there is no point in proving this) we have
\begin{align}
	&n_xd_x + n_yd_y + n_zd_z = 0 \\
	\implies &\frac{n_xd_x + n_yd_y}{n_zd_z} + 1 = 0 \\
	\implies &\frac{n_x}{n_z} \frac{d_x}{d_z} + \frac{n_y}{n_z}\frac{d_y}{d_z} + 1 = 0 \\
	\implies &\frac{d_x}{d_z} = - \frac{n_z}{n_x} - \frac{n_zn_yd_y}{n_xn_zd_z} \\
	\implies &\frac{d_x}{d_z} = - \frac{n_z}{n_x} - \frac{n_yd_y}{n_zd_z}
\end{align}
We can substitute this into what we had in part 1, i.e.
\begin{align}
	x_v &= \frac{fd_x + p_xd_x}{d_z} = f \frac{d_x}{d_z} + p_x \\
	&= f \left(- \frac{n_z}{n_x} - \frac{n_yd_y}{n_zd_z} \right) + p_x \\
	&= -f \frac{n_z}{n_x	} - f \frac{n_yd_y}{n_zd_z} + p_x = (\dag)
\end{align}
also, 
\begin{align}
	y_v = \frac{fd_y + p_yd_z}{d_z} = f \frac{d_y}{d_z} + p_y \implies \frac{d_y}{d_z} = \frac{y_v - p_y}{f}
\end{align}
Then, 
\begin{align}
	y_v 
	= (\dag) 
	&= -f \frac{n_z}{n_x} - f\frac{n_y (y_v - p_y)}{n_z f} + p_x \\
	&= \frac{fn_z}{n_x} - \frac{n_yy_v}{n_z} + \frac{n_yp_y}{n_z} + p_x
\end{align}
from where we can simplify
\begin{align}
	&n_xn_zx_v = -fn_zn_z - n_yn_xy_v + n_xn_yp_y + n_xn_zp_x \\
	\implies &\underbrace{n_xn_z}_{\alpha}x_v + \underbrace{n_yn_x}_{\beta}y_v = \underbrace{-fn_zn_z + n_xn_yp_y + n_xn_zp_x}_{\text{constant independent of $d$}}
\end{align}
which has exactly the form that we want in the first place, and it illustrates the linear relationship between $x_v$ and $y_v$. 

\section{Projection} 
\subsection{Part 1} Please check my implementation in file \texttt{q3.py}, which is based on the starter code provided. I shall present the derivation used to solve for $X, Y$ inside the code. Recall that the projection matrix is defined as
\begin{equation}
	\mathrm{P}=\underbrace{\left[\begin{array}{ccc}
		f & 0 & p_{x} \\
		0 & f & p_{y} \\
		0 & 0 & 1
	\end{array}\right]}_{\text {intrinsics } K} \underbrace{\left[\begin{array}{cccc}
		1 & 0 & 0 & 0 \\
		0 & 1 & 0 & 0 \\
		0 & 0 & 1 & 0
	\end{array}\right]}_{\text {projection }} \underbrace{\underbrace{\left[\begin{array}{cc}
	\mathrm{R}_{3 \times 3} & 0_{3 \times 1} \\
	0_{1 \times 3} & 1
	\end{array}\right]}_{\text {rotation }} \underbrace{\left[\begin{array}{cc}
	\mathrm{I}_{3 \times 3} & \mathrm{~T}_{3 \times 1} \\
		0_{1 \times 3} & 1
	\end{array}\right]}_{\text {translation }}}_{\begin{bmatrix}
		R  &  t
	\end{bmatrix} \,\,= \,\, \begin{bmatrix}
		R  &  -c
	\end{bmatrix}}
\end{equation}
which is a known 3 by 4 real matrix. Let's now write out the relationship between what we have and what we want
\begin{equation}
	P\begin{bmatrix}
		X \\ Y \\ Z \\ 1
	\end{bmatrix} = \begin{bmatrix}
		p_{11} & p_{12} & p_{13} & p_{14} \\
		p_{21} & p_{22} & p_{23} & p_{24} \\
		p_{31} & p_{32} & p_{33} & p_{34}
	\end{bmatrix}\begin{bmatrix}
		X \\ Y \\ Z \\1
	\end{bmatrix} = \begin{bmatrix}
		p_{11}X + p_{12}Y + p_{13}Z + p_{14} \\
		p_{21}X + p_{22}Y + p_{23}Z + p_{24} \\
		p_{31}X + p_{32}Y + p_{33}Z + p_{34}
	\end{bmatrix} = \begin{bmatrix}
		 ax \\ ay \\ a
	\end{bmatrix}
\end{equation}
where $a, ax, ay, Z$ and all the $p_{\cdot}$ are known and $X, Y$ are unknowns. We can just use the first two rows to get the relationship
\begin{equation}
	\underbrace{\begin{bmatrix}
		p_{11} & p_{12} \\
		p_{21} & p_{22}
	\end{bmatrix}}_{\text{known}}\underbrace{\begin{bmatrix}
		X \\ Y
	\end{bmatrix}}_{\text{unknown}} = \underbrace{\begin{bmatrix}
		ax - p_{13}Z - p_{14} \\
		ay - p_{23}Z - p_{23}
	\end{bmatrix}}_{\text{known}}
\end{equation}
Also note that due to the choice of $Z$ axis is $(0, 0, -1)$, I had to flip $Y$ direction in my code. Figures \ref{fig:3.1}, \ref{fig:3.2}, \ref{fig:3.3} are screenshots of the output. Since they are 3D, it would be the best if you run the code and view it in your browser, but the screenshot can give you a rough idea without having to execute the code. 
\begin{figure}
	\center\includegraphics[width=\textwidth]{figs/q3.1}
	\caption{\label{fig:3.1} Input 1, screenshot of reconstruction}
\end{figure}
\begin{figure}
	\center\includegraphics[width=\textwidth]{figs/q3.2}
	\caption{\label{fig:3.2} Input 2, screenshot of reconstruction}
\end{figure}
\begin{figure}
	\center\includegraphics[width=\textwidth]{figs/q3.3}
	\caption{\label{fig:3.3} Input 3, screenshot of reconstruction}
\end{figure}






\end{document}
